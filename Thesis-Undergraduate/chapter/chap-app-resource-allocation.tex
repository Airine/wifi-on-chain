\chapter{Resource Allocation}
\label{chap:resource}

In this session, the resource allocation mechanism used by our system will be introduced, and the reason we select it will be discussed as follow\cite{yang_price_2013}.

\section{Kelly Mechanism}

In this section, we give some background about the Kelly mechanism and further generalize it by using an embedded price differentiation. In the next section, we will explore the properties of the generalized mechanisms.

We consider a set $N=|\mathscr{N}|>1$ of rational users bidding for a share of divisible resource of capacity $C$. We assume that more than one user compete for the resource, i.e., $N=|\mathscr{N}|>1$. Each user i has a valuation function $v_{i}(\cdot)$, where $v_{i}\left(d_{i}\right)$ defines the monetary utility to user i when she is given $d_{i}$ amount of the resource.

A common objective in resource allocation is to maximize the social welfare. Under our context, it is to maximize the sum of the valuations of all the users as the following optimization problem:

\begin{itemize}
\item max           $\sum_{i \in \mathcal{N}} v_{i}\left(d_{i}\right)$
\item subject to    $\sum_{i \in \mathcal{N}} d_{i} \leq C \quad$ and $\quad d_{i} \geq 0 \forall i \in \mathcal{N}$
\end{itemize}

We define the above convex and compact constraint set as
$$
\mathscr{D}=\left\{\mathbf{d} | \sum_{i \in \mathcal{N}} d_{i} \leq c, \text { and } d_{i} \geq 0, \forall i \in \mathcal{N}\right\}
$$
In the Kelly mechanism [1] , each user $i$ submits a bid $w_{i} \geq 0$, which equals the payment for obtaining a share $d_{i}$ of the resource. We denote $u_{i}$ as the utility of each user $i$, defined in a quasi-linear [12] environment as $u_{i}\left(d_{i}\right)=v_{i}\left(d_{i}\right)-w_{i}$
which is the valuation of the allocated resource $v_{i}\left(d_{i}\right)$ less the payment $w_{i}$. The Kelly mechanism allocates the full capacity
C among all users and the resource share $d_{i}$ of each user $i$ is proportional to her bid $w_{i} .$ Mathematically, given a nonzero bid vector $\mathbf{w}=\left(w_{1}, w_{2}, \ldots, w_{N}\right),$ the resource allocation vector $\mathbf{d}=\left(d_{1}, d_{2}, \ldots, d_{N}\right)$ is defined by
$$
d_{i}=D_{i}(\mathbf{w})=\frac{w_{i}}{\sum_{j=1}^{N} w_{j}} c, \quad \forall i \in \mathcal{N}
$$
where $D_{i}(\cdot)$ denotes the proportional allocation function for user $i$ under the Kelly mechanism.

As a result of the Kelly mechanism, each user is charged the same unit price of the resource $\mu$ such that $\mu d_{i}=w_{i}$ for all users. This implicit unit price $\mu$ can be calculated as
$$
\mu=\frac{\sum_{j=1}^{N} w_{j}}{c}
$$

\section{The Generalized Kelly Mechanism}

Rather than implementing a nondiscriminatory price $\mu$ under the Kelly mechanism, we consider a price differentiation among users. Our motivation of designing the price differentiation is to achieve different efficiency points for the social welfare defined as the objective function of (1). Under our generalization, we consider a strict positive price vector $\mathbf{p}=$ $\left(p_{1}, p_{2}, \ldots, p_{N}\right)$ as a parameter of the mechanism. Each user $i$ submits a bid $t_{i} \geq 0$ to compete for the resource and the allocation rule is the same proportional rule defined in Eq. (2):
$$
d_{i}=D_{i}(\mathbf{t})=\frac{t_{i}}{\sum_{j=1}^{N} t_{j}} c, \quad \forall i \in \mathcal{N}
$$
The difference of our generalization from the Kelly mechanism is that each user i pays $p_{i} t_{i}$ amount of money for $D_{i}(\mathbf{t})$ amount
of shared resource, and therefore, obtains a utility of $u_{i}(\mathbf{t}, \mathbf{p})=v_{i}\left(d_{i}\right)-p_{i} t_{i}=v_{i}\left(D_{i}(t)\right)-p_{t} t_{i}$
This generalized mechanism can be imagined as a process where users buy divisible tickets to compete for the resource. We denote $t_{i}$ as the number of tickets bought by user $i$ and $p_{i}$ as the monetary price of each ticket for user i. Like the Kelly mechanism, it fully allocates the capacity $C$ among all users and the resource share $d_{i}=D_{i}(\mathbf{t})$ to each user $i$ is proportional to the number of tickets bought: $t_{i .}$ Although we do not differentiate tickets in resource allocation, the unit ticket price to users could be different. In particular, the Kelly mechanism is a special case of the generalization where $\mathbf{p}=1$

Compared to the Kelly mechanism, the generalized mechanism achieves a similar virtual unit price $v$ in terms of tickets (measured as tickets per unit of resource) defined as $v=\frac{\sum_{j=1}^{N} t_{j}}{c}$
Consequently, the effective/real unit price for resource among users will be proportional to the price vector $\mathbf{p},$ because each user i's real price becomes $p_{i} v$ (measured as abstract monetary units per unit of resource). Notice that although a pre-determined price is assigned to each user, the generalized mechanism inherits the simplicity/scalability of the Kelly mechanism in two ways: ( 1) the strategy space of the mechanism is still simply one-dimensional; and ( 2 ) only a single virtual price feedback, i.e., $v$, is required to be sent to all users.