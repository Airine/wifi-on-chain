\chapter{Introduction}
\label{chp:introduction}

\section{Background}

Over the past few decades, wireless communications and networking have experienced an unprecedented growth. However, along with the growth of wireless network technology, the number of active users and the Big Data on the Internet also experienced a near-exponential growth. Which bring many challenges to wireless resource allocation, such as limited wireless resources, differ multi-user need and trad-offs among multi-user service objectives \cite{wang_blockchain-based_2019}. 

To overcame these challenges, dynamic resource allocation is used to improve the overall system performance and ensure individual Quality of Service (QoS), which concerns about bandwidth capacity allocation. In practice, one of the biggest challenges of these problems is the system naturally does not know the characteristics of users and their applications. Moreover, the users are often autonomous and selfish, and they may misreport their real needs to maximize their own utilities. 

To mitigate the above problems, resource pricing mechanisms have been introduced to manage resource allocation. In this work, we use generalized Kelly mechanism with price differentiation mechanism\cite{yang_price_2013} by R.M, which allows autonomous resource owners to apply different price differentiation schemes so as to achieve individual objectives, e.g., making trade-offs between user fairness and system efficiency. The generalized mechanism extends the flexibility of the Kelly mechanism and it is also adaptive and robust, since the mechanism depends on a congestion pricing principle and the allocations are implemented as Nash equilibrium solutions.

In order to avoid the security risks brought by running mentioned resource allocation mechanism on a central server, blockchain technology, a promising technology to address trust and security concerns, is apply to deploy the generalized Kelly mechanism as a smart contract. Which can be used to ensure the related transactions are unmodifiable and undeniable, and make the overall service open and transparent.

\textbf{In this work, a bidding based wireless resource allocation system is proposed, which consist of a "serverless" web application, a a smart contract and private blockchain node.} For quick verification purpose, the bandwidth control is implemented by running a python script to maintain a IP table of active users. To make the system more closer to industrial use, I will try to use a captive portal to do the wireless network control and monitoring.

\section{Related Works}

\textbf{Resource allocation mechanism are wide need over computing resources}, e.g., computing services, network resources and content streaming services, have been studied extensively during the last decade. And due to the characteristics of blockchain, scholars have conducted related research on the combination of blockchain technology with cloud computing, fog computing and edge computing, including research with the Internet of Things, access control technology and other related fields. This thesis focuses on resource allocation in a edge computing environment. Both the blockchain node and captive portal run on a router or a Respberry Pi.

WANG H, et al.\cite{wang_blockchain-based_2019} proposed a resource contribution model between the fog node and cloud or users, which practices the reward and punishment mechanism of the blockchain to boost the fog nodes to contribute resources actively. With the behavior of the fog node in contributing resources are stored in the blockchain, a transparent, open and tamper-free service evaluation index forms. In their model, the designed reward and punishment mechanism is very similar with the mining activity of miners.

LIU M, et al.\cite{liu_distributed_2019} proposed a blockchain-based video streaming system, aiming to build decentralized peer-to-peer networks with flexible monetization mechanisms for video streaming services. In this thesis, the author designed a novel blockchain-based framework with adaptive block size for video streaming with mobile edge computing.

% ABDELLATIF K, ABDELMOUTTALIB C.  TO BE DECIDED.

\section{Motivation and Contributions}

\textbf{The main motivation of this work is to show how blockchain system could help with resouce allocation,} how  smart contract could provide bidding or pricing based system avoid security risks and achieve indeed transparency and unmodifiable properties. And theoretically, the bandwidth capacity of wireless network, an example wireless resource, can be easily replace with any other wireless resource with few modification. 

\textbf{The contributions of this work are} 1. proposed and designed a bidding/pricing based resource allocation system on smart contract, 2. implemented a practical system focus on wireless bandwidth capacity allocation, 3. built a private PoA blockchain network and plan to evaluate the performance difference with PoW network.

\section{Thesis Structure}

This thesis is organised as follows:
In Chapter. 2, the model design of the overall system is shown.
Then, the generalized Kelly mechanism is introduced in detail in Chapter. 3. 
In Chapter. 4, the blockchain related terms such as "Smart Contracts", "Consensus Algorithm" and "DApp" are introduced in detail. 
Subsequently, my implementation of the system and challenging problems I faced during implementation are presented in Chapter. 5. 
At last, some performance evaluation and comparison are made in Chapter. 6.

