\begin{cnabstract}
基于竞价的无线资源分配可以根据用户的需求和网络系统带宽的使用情况来计算网络带宽的价格,并通过集中竞价通过用户的价格参数进行带宽资源的协商和分配。而本文的资源分配思想是基于Kelly机制,它允许用户根据自己的需求获取带宽。通常,竞价系统运行在集中式服务器上,该系统易于设计,但可能具有许多隐藏的安全风险,包括透明性,数据互操作性风险,网络隐私和安全漏洞。

区块链是一种可以轻松解决这些风险的技术,可用于确保交易不可修改且不可否认。另外,由于我们在路由器的边缘部署了竞价和带宽控制系统,因此,在将系统应用于由许多路由器组成的企业网络时,区块链也显示出极大的弹性。在本文中,我们建立了具有新的权威证明机制的私有区块链网络,该机制可以在路由器等边缘设备中以低能耗达成共识。平均而言,与传统的区块链系统相比,采用新型共识算法的区块链系统的响应时间缩短了xx%,电池消耗的电量减少了xx%。

总之,这片文章提出了一个基于区块链智能合约实现的竞价无线资源分配系统。为了使得模型更贴近真实场景,本文以公共区域的Wi-Fi认证与带宽分配系统作为特定的无线资源进行分配。但是,由于不同类型的无线资源间具有相似的用户场景和很高的结构相似度,本文提出并实现的系统可经过较少的改变使其适用于别的无线资源如:4G和5G等等。

\keywords{资源分配, 区块链,智能合约,共识算法}
\end{cnabstract}

\begin{enabstract}

The bidding based wireless resource allocation calculates the price of network bandwidth according to the user's demand and the usage of network system bandwidth and carries out the negotiation and allocation of bandwidth resources through the user's price parameters through centralized bidding. And the resource allocation idea of this paper is based on the Kelly mechanism\cite{yang_price_2013}, which allows users to acquire bandwidth according to their demand. Normally, a bidding system is running on a centralized server, which is easy to design but may have many hidden security risks including transparency, risks of data interoperability, network privacy and security vulnerabilities.

Blockchain, a technology that can easily solve these risks , can be used to ensure transactions are unmodifiable and undeniable. Also, as we deploy the bidding and bandwidth control system on the edge - the router, blockchain also shows great elasticity while applying the system on an enterprise network consisting of many routers. In this paper, we establish a private blockchain network with a new Proof of Authority mechanism, which can reach consensus with low energy consumption in edge devices like a router. On average, the blockchain system with the new consensus algorithm shows xx\% less response time and xx\% less battery power consumed when compared with traditional blockchain systems.

In short, this paper proposes a bidding based wireless resource allocation system using blockchain smart contracts. In order to make the model closer to the real scene, this paper uses the Wi-Fi authentication and bandwidth allocation system in the public area as a specific wireless resource for allocation. However, due to similar user scenarios and high structural similarity between different types of wireless resources, the system proposed and implemented in this paper can be adapted to other wireless resources such as 4G and 5G with few changes.

Keywords: resource allocation, blockchain, smart contract, consensus algorithm


\enkeywords{resource allocation, blockchain, smart contract, consensus algorithm}
\end{enabstract}
